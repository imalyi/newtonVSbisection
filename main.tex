\documentclass[12pt]{article}
\usepackage{graphicx}
\usepackage{indentfirst}
\usepackage{polski}
\usepackage{amsmath}
\usepackage{amssymb}
\usepackage{dirtree}
\usepackage{listings}
\usepackage{hyperref}
\usepackage{minted}
\usepackage{xcolor}
\usepackage{subcaption}
\definecolor{LightGray}{gray}{0.9}
\hypersetup{
    colorlinks=true,
    linkcolor=blue}
\newtheorem{example}{Przykład}

\title{Dokumentacja projektu zespołowego nr 2}
\author{Anna Ćwiklińska, Krystian Gronkowski, Ihor Malyi}
\date{Marzec 2023}

\begin{document}
\lstset{basicstyle=\ttfamily, columns=fullflexible, upquote=true}
\renewcommand{\lstlistingname}{Listing}

\maketitle

\section{Treść zadania}
Pobrać od użytkownika żądaną dokładność $0<\varepsilon<1$ oraz przedział $[a,b]$, w~którym szukamy pierwiastka równania $$\ln(x^2) - \sin(x) - 2 = 0.$$

Porównać liczbę kroków potrzebnych metodzie siecznych Newtona, by osiągnąć dokładność $\varepsilon$ z~liczbą kroków dla metody bisekcji dla kilku wybranych przedziałów zawierających dokładnie jeden pierwiastek. Znaleźć wszystkie pierwiastki równania z~dokładnością $10^{-8}$.

\section{Teoretyczny opis metody}


\subsection{Metoda siecznych Newtona}
Metoda siecznych Newtona to jedna z~metod numerycznych służących do znajdowania miejsc zerowych funkcji. Metoda ta polega na przybliżaniu pierwiastka równania poprzez konstrukcję linii siecznej przechodzącej przez dwa punkty na wykresie funkcji i~wyznaczeniu jej przecięcia z~osią OX. Następnie punkt ten jest wykorzystywany w~kolejnej iteracji metody, aż do uzyskania dostatecznie dokładnego wyniku.

W~metodzie siecznych Newtona w~każdej iteracji przybliżenie pierwiastka równania jest obliczane jako przecięcie linii siecznej z~osią OX, zdefiniowanej jako:

$$x_{n+1} = x_n - \frac{f(x_n)(x_n-x_{n-1})}{f(x_n)-f(x_{n-1})}$$

gdzie $x_n$ i $x_{n-1}$ są kolejnymi przybliżeniami pierwiastka równania, a~$f(x)$ jest funkcją, której pierwiastka szukamy.

\begin{example}
Rozważmy równanie $x^2 - 2 = 0$, dla którego pierwiastkiem jest $\sqrt{2}$. Chcemy znaleźć pierwiastek tego równania z~dokładnością $\varepsilon = 10^{-4}$.

Zacznijmy od wybrania dwóch początkowych przybliżeń pierwiastka równania, na przykład $x_0 = 1$ i~$x_1 = 2$. Następnie stosujemy wzór metody siecznych Newtona:

$$x_{n+1} = x_n - \frac{f(x_n)(x_n-x_{n-1})}{f(x_n)-f(x_{n-1})}$$

Dla $n=0$ mamy:
\begin{align*}
    &f(x_0) = 1^2 - 2 = -1\\
    &f(x_1) = 2^2 - 2 = 2\\
    &x_{2} = 2 - \frac{2\cdot (2-1)}{2-(-1)} = \frac{5}{3} \approx 1.6667
\end{align*}


Dla $n=1$ mamy:
\begin{align*}
    &f(x_1) = 2^2 - 2 = 2\\
    &f(x_2) = \left(\frac{5}{3}\right)^2 - 2 \approx -0.1111\\
    &x_{3} = \frac{5}{3} - \frac{(-0.1111) \cdot \left(\frac{5}{3}-2\right)}{(-0.1111)-2} \approx 1.4143
\end{align*}

Teraz obliczamy różnicę między kolejnymi przybliżeniami:
$$|x_{3} - x_{2}| = |1.4143 - 1.6667| \approx 0.2524$$

Ponieważ wartość ta jest mniejsza niż zadana dokładność $\varepsilon = 10^{-4}$, kończymy obliczenia i~zwracamy ostatnie przybliżenie $x_3 = 1.4143$ jako przybliżenie rozwiązania.
\end{example}

\subsection{Metoda bisekcji}
Dla funkcji ciągłej $f$ określonej na przedziale domkniętym $[a,b]$ oraz spełniającej warunek $f(a) \cdot f(b) < 0$ (czyli funkcja zmienia znak na tym przedziale) możemy znaleźć pierwiastek równania $f(x) = 0$ poprzez wykonanie kroków algorytmu:

\begin{enumerate}
\item Ustawiamy $a_0 = a$, $b_0 = b$, $i = 0$.
\item Obliczamy $c_i = \frac{a_i + b_i}{2}$.
\item Jeśli $f(c_i) = 0$, kończymy obliczenia i zwracamy $c_i$ jako rozwiązanie.
\item Jeśli $f(a_i) \cdot f(c_i) < 0$, ustawiamy $a_{i+1} = a_i$ oraz $b_{i+1} = c_i$.
\item W przeciwnym przypadku, gdy $f(b_i) \cdot f(c_i) < 0$, ustawiamy $a_{i+1} = c_i$ oraz $b_{i+1} = b_i$.
\item Jeśli osiągnięto zadany poziom dokładności (czyli $|b_i - a_i| < \epsilon$, gdzie $\epsilon$ to ustalona wartość dokładności), zwracamy $c_i$ jako rozwiązanie.
\item W przeciwnym przypadku, ustawiamy $i = i + 1$ i wracamy do kroku 2.
\end{enumerate}

Algorytm kończy się, gdy zostanie osiągnięty poziom dokładności lub zostanie wykonana maksymalna liczba iteracji. Warto zauważyć, że metoda bisekcji zawsze znajduje pierwiastek równania $f(x) = 0$ na danym przedziale $[a,b]$, pod warunkiem, że funkcja $f$ jest ciągła i zmienia znak na tym przedziale.

Możemy również zdefiniować błąd oszacowania wartości pierwiastka jako $|c_i - c_{i-1}|$, gdzie $c_i$ to wartość pierwiastka obliczona w $i$-tej iteracji. Błąd ten maleje monotonicznie z każdą kolejną iteracją.

\begin{example}
Dana jest funkcja $F(x) = x^3 - 3x^2 - 2x + 5$ oraz przedział $[a, b] = [1,2]$.
W celu znalezienia pierwiastka $F(x) = 0$ z dokładnością $\varepsilon = 0.1$ zostanie użyta metoda bisekcji.

\begin{enumerate}
    \item Iteracja:\\
    \begin{align*}
        &x_1=1,\ x_2=2\\
        &x=\frac{x_1+x_2}{2}=\frac{1+2}{2}=1.5\\
        &y=F(x)=F(1.5)=-1.375\\
        &|F(x)|>\varepsilon\\
        &y_1=F(x_1)=F(1)=1\\
        &y\cdot y_1=-1.375<0\rightarrow x_2=x=1.5
    \end{align*}
    
    \item Iteracja:\\
    \begin{align*}
        &x_1=1,\ x_2=1.5\\
        &x=\frac{x_1+x_2}{2}=\frac{1+1.5}{2}=1.25\\
        &y=F(x)=F(1.25)=-0.234\\
        &|F(x)|>\varepsilon\\
        &y_1=F(x_1)=F(1)=1\\
        &y\cdot y_1=-0.234<0\rightarrow x_2=x=1.25
    \end{align*}

    
    \item Iteracja:\\
    \begin{align*}
        &x_1=1,\ x_2=1.25\\
        &x=\frac{x_1+x_2}{2}=\frac{1+1.25}{2}=1.125\\
        &y=F(x)=F(1.125)=0.376\\
        &|F(x)|>\varepsilon\\
        &y_1=F(x_1)=F(1)=1\\
        &y\cdot y_1=1.125>0\rightarrow x_1=x=1.125
    \end{align*}
    
    \item Iteracja:\\
    \begin{align*}
        &x_1=1.125,\ x_2=1.25\\
        &x=\frac{x_1+x_2}{2}=\frac{1.125+1.25}{2}=1.1875\\
        &y=F(x)=F(1.1875)=0.069\\
        &|F(x)|<\varepsilon\\
    \end{align*}
\end{enumerate}

Zatem $x=1.1875$
\end{example}


\end{document}
